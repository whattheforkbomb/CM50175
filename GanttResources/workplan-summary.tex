The workplan comprises \arabic{WPLast} work packages, of which \wpfullref{WPUser} is the starting point, through which we start to elicit early requirements from a range of stakeholders aiming to establish connections between what stakeholders (rights holders, legal experts, individuals) seek in terms of functionality and what technology can deliver.  For the latter, at the same time \wpfullref{WPCrypto} and \wpfullref{WPTools} are spreading and deepening the team's knowledge of the state of the art in blockchain technology (including dialogue with the companies associated with DC and HEMM who are actively exploring its application) and policy reasoning (Bath and Bristol).  This first phase (Q1) sees the recruitment of RAs for \wpref{\WPCrypto} and \wpref{\WPTools}.  The RA for \wpref{\WPCloud} will be recruited in Q6 (see workplan).  The RAs for \wpfullref{WPMaths} can be subcontracted flexibly/part-time from Bath IMI as and when needed.
%
The core of the work programme rests on three activities:
\begin{inparaenum}[(i)]
\item research on technical aspects (\wpref{\WPCrypto}, \wpref{\WPTools}, \wpref{\WPCloud})
\item research on technical and human aspects of deployment (\wpref{\WPCloud}, \wpref{\WPUser})
\item research on human (individual and societal) aspects of uptake (\wpref{\WPMaths}, \wpref{\WPUser}).
\end{inparaenum}
% This phase depends on significant interaction between the technical WPs, where \wpref{\WPCrypto} and \wpref{\WPTools} deliver new cryptographic results and reasoning tools, while \wpref{\WPCloud} provides both an experimental deployment environment and new technical challenges for security.  At the same time \wpfullref{WPMaths} will begin modelling diffusion so we can understand what aids and hinders uptake, using the results of the initial surveys from \wpref{\WPUser} from phase one, refined by follow-up surveys on the tools being developed in this phase.  A crucial feature of the work plan is continuing engagement during the core research phase with the various user groups brought by DC and HEMM to allow frequent (re-)evaluation of requirements, deliverables and priorities.  This process will be reinforced by the pathway to impact activities being hosted by DC during the project, and which should lead in due course to the project outcomes being integrated into the architecture of DC's Trust Framework Initiative.  The interactions between technical, human and societal aspects will be subject to careful management and timing to ensure maximum impact.
