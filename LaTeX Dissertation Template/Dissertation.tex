% Modified by Alessio Guglielmi, 3 February 2021

\documentclass[12pt,a4paper]{report}
% This document template assumes you will use pdflatex.  If you are using
% latex and dvipdfmx to translate to pdf, insert dvipdfmx into the options.

\usepackage{Bath-CS-Dissertation}

\title{\bf $\langle$Dissertation Title$\rangle$}
\author{$\langle$Student Name$\rangle$}
\date{MSc $\langle$EngD$\rangle$ in $\langle$Programme$\rangle$\\ 
                                 % E.g.: Bachelor of Science in Computer Science
                                 %       Master of Science in Data Science
      The University of Bath\\
      $\langle$Academic Year$\rangle$}

%%%%%%%%%%%%%%%%%%%%%%%%%%%%%%%%%%%%%%%%%%%%%%%%%%%%%%%%%%%%%%%%%%%%%%%%%%%%%%%%
%%%%%%%%%%%%%%%%%%%%%%%%%%%%%%%%%%%%%%%%%%%%%%%%%%%%%%%%%%%%%%%%%%%%%%%%%%%%%%%%
%%%%%%%%%%%%%%%%%%%%%%%%%%%%%%%%%%%%%%%%%%%%%%%%%%%%%%%%%%%%%%%%%%%%%%%%%%%%%%%%
\begin{document}

\hypersetup{pageanchor=false}

% Set this to the language you want to use in your code listings (if any)
\lstset{language=Java,breaklines,breakatwhitespace,basicstyle=\small}

\setcounter{page}{0}
\pagenumbering{roman}

\maketitle
\newpage

% Set this to the number of years consultation prohibition, or 0 if no limit
\consultation{0}
\newpage

\declaration{$\langle$Dissertation Title$\rangle$}{$\langle$Student Name$\rangle$}
\newpage

\hypersetup{pageanchor=true}

\abstract
$\langle$The abstract should appear here. An abstract is a short paragraph describing the aims of the project, what was achieved and what contributions it has made.$\rangle$
\newpage

\tableofcontents
\newpage

\listoffigures
\newpage

\listoftables
\newpage

%%%%%%%%%%%%%%%%%%%%%%%%%%%%%%%%%%%%%%%%%%%%%%%%%%%%%%%%%%%%%%%%%%%%%%%%%%%%%%%%
%%%%%%%%%%%%%%%%%%%%%%%%%%%%%%%%%%%%%%%%%%%%%%%%%%%%%%%%%%%%%%%%%%%%%%%%%%%%%%%%
\chapter*{Acknowledgements}

Add any acknowledgements here.

\newpage
\setcounter{page}{1}
\pagenumbering{arabic}

%%%%%%%%%%%%%%%%%%%%%%%%%%%%%%%%%%%%%%%%%%%%%%%%%%%%%%%%%%%%%%%%%%%%%%%%%%%%%%%%
%%%%%%%%%%%%%%%%%%%%%%%%%%%%%%%%%%%%%%%%%%%%%%%%%%%%%%%%%%%%%%%%%%%%%%%%%%%%%%%%
\chapter{Introduction}

This is the introductory chapter.

%%%%%%%%%%%%%%%%%%%%%%%%%%%%%%%%%%%%%%%%%%%%%%%%%%%%%%%%%%%%%%%%%%%%%%%%%%%%%%%%
\section{Example Section}

Like all chapters, it will have a number of sections \dots

%-------------------------------------------------------------------------------
\subsection{Example Subsection}
\dots\ and subsections \dots

\subsubsection{Example Sub-subsection}
\dots\ and sub-subsections.

\section[Short Section Title]{Another Section With a Long Title and Whose Title Is Abbreviated in the Table of Contents}

%-------------------------------------------------------------------------------
\begin{table}[htb]
\caption{An example table}
\bigskip
\begin{center}
\label{Example-Table}
\begin{tabular}{|l|l|}
\hline
Items & Values \\
\hline
\hline
Item 1 & Value 1 \\
Item 2 & Value 2 \\
\hline
\end{tabular}
\end{center}
\end{table}

Another section, just for good measure. You can reference a table, figure or equation using \verb|\ref|, just like this reference to Table \ref{Example-Table}.

%%%%%%%%%%%%%%%%%%%%%%%%%%%%%%%%%%%%%%%%%%%%%%%%%%%%%%%%%%%%%%%%%%%%%%%%%%%%%%%%
\section{Example Lists}

%-------------------------------------------------------------------------------
\subsection{Enumerated}

\begin{enumerate}
\item Example enumerated list:
  \begin{itemize}
  \item a nested enumerated list item;
  \item and another one.
  \end{itemize}
\item Second item in the list.
\end{enumerate}

%-------------------------------------------------------------------------------
\subsection{Itemised}

\begin{itemize}
\item Example itemised list.
  \begin{itemize}
  \item A nested itemised list item.
  \end{itemize}
\item Second item in the list.
\end{itemize}

%-------------------------------------------------------------------------------
\subsection{Description}

\begin{description}
\item[Item 1]First item in the list.
\item[Item 2]Second item in the list.
\end{description}

%%%%%%%%%%%%%%%%%%%%%%%%%%%%%%%%%%%%%%%%%%%%%%%%%%%%%%%%%%%%%%%%%%%%%%%%%%%%%%%%
%%%%%%%%%%%%%%%%%%%%%%%%%%%%%%%%%%%%%%%%%%%%%%%%%%%%%%%%%%%%%%%%%%%%%%%%%%%%%%%%
\chapter{Literature and Technology Survey}

This is the chapter for your Literature and Technology Survey.

You will wish to cite authors like \citep{Brun:03:Atomic-C:oz} or \citep{Stra:17:Combinat:qf}. Alternate
commands are used to cite \citet{Brun:03:Atomic-C:oz} as a noun, or add text to the citation, \citep[\emph{e.g.},][]{Stra:17:Combinat:qf}.

%% NOTE Replace the following with chapters that are appropriate for your
%%      style of project.  It is unlikely these will fit your project perfectly.

%%%%%%%%%%%%%%%%%%%%%%%%%%%%%%%%%%%%%%%%%%%%%%%%%%%%%%%%%%%%%%%%%%%%%%%%%%%%%%%%
%%%%%%%%%%%%%%%%%%%%%%%%%%%%%%%%%%%%%%%%%%%%%%%%%%%%%%%%%%%%%%%%%%%%%%%%%%%%%%%%
\chapter{Requirements}

If you are doing a primarily software development project, this is the chapter in which you review the requirements decisions and critique the requirements process.

%%%%%%%%%%%%%%%%%%%%%%%%%%%%%%%%%%%%%%%%%%%%%%%%%%%%%%%%%%%%%%%%%%%%%%%%%%%%%%%%
%%%%%%%%%%%%%%%%%%%%%%%%%%%%%%%%%%%%%%%%%%%%%%%%%%%%%%%%%%%%%%%%%%%%%%%%%%%%%%%%
\chapter{Design}

This is the chapter in which you review your design decisions at various
levels and critique the design process.

%%%%%%%%%%%%%%%%%%%%%%%%%%%%%%%%%%%%%%%%%%%%%%%%%%%%%%%%%%%%%%%%%%%%%%%%%%%%%%%%
%%%%%%%%%%%%%%%%%%%%%%%%%%%%%%%%%%%%%%%%%%%%%%%%%%%%%%%%%%%%%%%%%%%%%%%%%%%%%%%%
\chapter{Implementation and Testing}

This is the chapter in which you review the implementation and testing decisions and issues, and critique these processes.

Code can be output inline using \verb@\lstinline|some code|@.  For example, this code is inline: \lstinline|public static int example = 0;| (we have used the character \verb@|@ as a delimiter, but any non-reserved character not in the code text can be used.)

Code snippets can be output using the \verb|\begin{lstlisting} ... \end{lstlisting}|
environment with the code given in the environment. For example, consider listing \ref{Example-Code}, below.

\begin{lstlisting}[breaklines,breakatwhitespace,caption={Example code},label=Example-Code]
public static void main() {

  System.out.println("Hello World");

}
\end{lstlisting}

Code listings are produced using the package `listings'.  This has many useful options, so have a look at the package documentation for further ideas.

%%%%%%%%%%%%%%%%%%%%%%%%%%%%%%%%%%%%%%%%%%%%%%%%%%%%%%%%%%%%%%%%%%%%%%%%%%%%%%%%
%%%%%%%%%%%%%%%%%%%%%%%%%%%%%%%%%%%%%%%%%%%%%%%%%%%%%%%%%%%%%%%%%%%%%%%%%%%%%%%%
\chapter{Results}

This is the chapter in which you review the outcomes, and critique the outcomes process. You may include user evaluation here too.

%%%%%%%%%%%%%%%%%%%%%%%%%%%%%%%%%%%%%%%%%%%%%%%%%%%%%%%%%%%%%%%%%%%%%%%%%%%%%%%%
%%%%%%%%%%%%%%%%%%%%%%%%%%%%%%%%%%%%%%%%%%%%%%%%%%%%%%%%%%%%%%%%%%%%%%%%%%%%%%%%
\chapter{Conclusions}

%%
%% Now we are back to the standard project contents that you should include
%%

This is the chapter in which you review the major achievements in the light of your original objectives, critique the process, critique your own learning and identify possible future work.

%%%%%%%%%%%%%%%%%%%%%%%%%%%%%%%%%%%%%%%%%%%%%%%%%%%%%%%%%%%%%%%%%%%%%%%%%%%%%%%%
%%%%%%%%%%%%%%%%%%%%%%%%%%%%%%%%%%%%%%%%%%%%%%%%%%%%%%%%%%%%%%%%%%%%%%%%%%%%%%%%
\bibliography{ExampleBibFile}

%%%%%%%%%%%%%%%%%%%%%%%%%%%%%%%%%%%%%%%%%%%%%%%%%%%%%%%%%%%%%%%%%%%%%%%%%%%%%%%%
%%%%%%%%%%%%%%%%%%%%%%%%%%%%%%%%%%%%%%%%%%%%%%%%%%%%%%%%%%%%%%%%%%%%%%%%%%%%%%%%
\appendix

%%
%% Use the appendix for major chunks of detailed work, such as these. Tailor
%% these to your own requirements
%%

%%%%%%%%%%%%%%%%%%%%%%%%%%%%%%%%%%%%%%%%%%%%%%%%%%%%%%%%%%%%%%%%%%%%%%%%%%%%%%%%
%%%%%%%%%%%%%%%%%%%%%%%%%%%%%%%%%%%%%%%%%%%%%%%%%%%%%%%%%%%%%%%%%%%%%%%%%%%%%%%%
\chapter{Design Diagrams}

%%%%%%%%%%%%%%%%%%%%%%%%%%%%%%%%%%%%%%%%%%%%%%%%%%%%%%%%%%%%%%%%%%%%%%%%%%%%%%%%
%%%%%%%%%%%%%%%%%%%%%%%%%%%%%%%%%%%%%%%%%%%%%%%%%%%%%%%%%%%%%%%%%%%%%%%%%%%%%%%%
\chapter{User Documentation}

%%%%%%%%%%%%%%%%%%%%%%%%%%%%%%%%%%%%%%%%%%%%%%%%%%%%%%%%%%%%%%%%%%%%%%%%%%%%%%%%
%%%%%%%%%%%%%%%%%%%%%%%%%%%%%%%%%%%%%%%%%%%%%%%%%%%%%%%%%%%%%%%%%%%%%%%%%%%%%%%%
\chapter{Raw Results Output}

%%%%%%%%%%%%%%%%%%%%%%%%%%%%%%%%%%%%%%%%%%%%%%%%%%%%%%%%%%%%%%%%%%%%%%%%%%%%%%%%
%%%%%%%%%%%%%%%%%%%%%%%%%%%%%%%%%%%%%%%%%%%%%%%%%%%%%%%%%%%%%%%%%%%%%%%%%%%%%%%%
\chapter{Code}

%% NOTE For this to typeset correctly, ensure you use the pdflatex
%%      command in preference to the latex command.  If you do not have
%%      the pdflatex command, you will need to remove the landscape and
%%      multicols tags and just make do with single column listing output

\begin{landscape}
\begin{multicols}{2}
\section{File: yourCodeFile.java}
\lstinputlisting[basicstyle=\scriptsize]{yourCodeFile.java}
\end{multicols}
\end{landscape}

\end{document}
