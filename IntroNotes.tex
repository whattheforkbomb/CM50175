%% Intro

% What 
% |- Will be developing and evaluating a system that permits the user to interact with their phone hands-free / one-handed
%
% Why
% |- Situationally Induced Impairments and Disabilities (SIIDs)
% |  | Scenarios where user is suffering some affliction induced by the environment they're operating within, and as such
% |  |  they are either unable to or have less ability to perform a task using their phone
% |
% |- One-Handed use of a phone is common [Is this a valid statement?]
% |  | For simple tasks, you don't need to use more than one input via the touch-screen (e.g. just thumb)
% |  | Larger phone screen sizes impact usability of single-handed input (via thumb) [thumb reachability studies]
% |  |- Phone/grip shifting
% |- Issues of 1 handed usage (thumb reachability)


% Look at https://faculty.washington.edu/wobbrock/pubs/Wobbrock-2015.pdf


% Phone screens are getting larger
%  screen usability goes down with screen-size, unless two-handed interaction is enforced, or device supported externally
%  people prefer one-handed, even when not encumbered

% Why use head as input
%   One solution to the issue is to provide additional input modalities than just touch, such as using the head
%   Why use head vs thumb, voice, Additional haptic / non-screen touch interactions

% Why adapt interface to context
%   Another solution is to adapt the interface to better suit the user's current context

% We want to develop a system that adapts the interface based on user context, primarily driven by user perspective, and which supports the use of head gestures to aid in interaction with larger screens, one-handed or hands-free

%% QUOTES & NOTES %%

% @inproceedings{xuesheng2018research,
%   title={Research on the Development Law of Smart Phone Screen based on User Experience},
%   author={Xuesheng, Pei and Yang, Wang},
%   booktitle={MATEC Web of Conferences},
%   volume={176},
%   pages={04006},
%   year={2018},
%   organization={EDP Sciences}
% }
\cite{xuesheng2018research} Modern mobile device screens 


%% SIIDs
% @article{yesiladaweb,
%   title={Web Accessibility},
%   author={Yesilada, Yeliz and Harper, Simon},
%   publisher={Springer}
% }
described by \citeauthor{yesiladaweb} as "situations, contexts, or environments that negatively affect the abilities of people interacting with technology" (\citeyear{yesiladaweb})\\

% @inproceedings{sarsenbayeva2017challenges,
%   title={Challenges of situational impairments during interaction with mobile devices},
%   author={Sarsenbayeva, Zhanna and van Berkel, Niels and Luo, Chu and Kostakos, Vassilis and Goncalves, Jorge},
%   booktitle={Proceedings of the 29th Australian Conference on Computer-Human Interaction},
%   pages={477--481},
%   year={2017}
% }
\cite{sarsenbayeva2017challenges} suggest further defining Severely Constraining Situational Impairments” (SCSI), wherein the the ability to input to the device is restricted.\\

% @article{karlson2006understanding,
%   title={Understanding single-handed mobile device interaction},
%   author={Karlson, Amy K and Bederson, Benjamin B and Contreras-Vidal, J},
%   journal={Handbook of research on user interface design and evaluation for mobile technology},
%   volume={1},
%   pages={86--101},
%   year={2006},
%   publisher={National Research Council of Canada Institute for Information Technology Hershey}
% }
% Can possibly use as example that one-handed took off with touch-screens, and connects the SIID section with general one-handed use
\cite{karlson2006understanding} undertook a couple studies to investigate both the support for one-handed use offered by mobile devices, and the scenarios wherein users would operate the device single handedly.\\
From their field study (23 people) they observed a correlation between one-handed interaction and the user's movement. If the user was mobile or standing, they were much more likely to be using the phone one-handed as compared to when they were sat down.\\
In their study of 135 people, they found that 66\% of participants preferred one-handed use, regardless of whether they were encumbered (e.g. suffering from SIID), resorting to two-handed use only when the interface requires it.\\

%% One-handed usage
% @article{hoober2013users,
%   title={How do users really hold mobile devices},
%   author={Hoober, Steven},
%   journal={Uxmatters (http://www. uxmatter. com). Published: Feburary},
%   volume={18},
%   pages={2327--4662},
%   year={2013}
% }
From a sample of 1334 observations (Unclear on number of people), \cite{hoober2013users} observed that 49\% of mobile device interaction was one-handed, with 36\% being one-handed with the second hand cradling the phone, and the last 15\% being two-handed (wherein both thumbs were in use).\\
They go o0n to discuss the effective touch-screen reach the user can obtain with each method, and how users switch between these different modes based on adapting to completing different tasks (primarily if completing a task requires access to an unreachable part of the phone that they can't access in the current 'stance')\\
% 2013 (9yrs old), still relevant enough? 

% @misc{samsung2021onehand,
%     title={Using One Handed Mode on my Samsung Phone},
%     author={Samsung},
%     year={2021},
%     url={https://www.samsung.com/au/support/mobile-devices/using-one-handed-mode/},
%     urldate = {2022/04/01}
% }
Given how prevalent one0handed use is, Samsung \citep{samsung2021onehand} (and other platforms?) have a dedicated one-handed mode, which effectively shrinks the screen towards the corner which has the most reach for the user's thumb.\\
However this has the downside of shrinking the screen content, and reduces the accuracy of the thumb input (as where the thumb interacts with the screen is now effectively increased)\\
% Should this be moved to lit review as this is an implementation, or can this be left here as a driving force, e.g. want to do better?

%% Reachability
% @inproceedings{le2018fingers,
%   title={Fingers' Range and Comfortable Area for One-Handed Smartphone Interaction Beyond the Touchscreen},
%   author={Le, Huy Viet and Mayer, Sven and Bader, Patrick and Henze, Niels},
%   booktitle={Proceedings of the 2018 CHI Conference on Human Factors in Computing Systems},
%   pages={1--12},
%   year={2018}
% }
Study with 4 phones with increasing screen size, investigating the areas of the screen which are obtainable with each digit (Screen for thumb, phone back for fingers)\\
As expected, larger screen sizes afforded less coverage for the thumb, in-particular reaching the upper parts of the screen, and in some-cases the side of the screen opposite to the thumb\\
However this study did not ensure grip was consistent between participants, or that participants keep a constant grip. As such it isn't clear if some observed reachability is due to grip, hand-size, or adjusted grips.\\

% @article{choi2020effects,
%   title={Effects of smartphone size and hand size on grip posture in one-handed hard key operations},
%   author={Choi, Younggeun and Yang, Xiaopeng and Park, Jangwoon and Lee, Wonsup and You, Heecheon},
%   journal={Applied Sciences},
%   volume={10},
%   number={23},
%   pages={8374},
%   year={2020},
%   publisher={Multidisciplinary Digital Publishing Institute}
% }
% Possibly reference with regards to how larger phone afford particular grips, that may reduce reachability due to comfort (easy and comfortable to maintain grip) and security (believe won't drop phone)

% Wanting to use the head, and why
% Don't need to discuss particular examples, unless works want to build upon.
% No extensive analysis/review, just highlighting reasons and can then perform review & analysis within the Lit Review

% What are the research questions